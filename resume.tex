%%%%%%%%%%%%%%%%%%%%%%%%%%%%%%%%%%%%%%%%%
% Medium Length Professional CV
% LaTeX Template
% Version 2.0 (8/5/13)
%
% This template has been downloaded from:
% http://www.LaTeXTemplates.com
%
% Original author:
% Rishi Shah 
%
% Important note:
% This template requires the resume.cls file to be in the same directory as the
% .tex file. The resume.cls file provides the resume style used for structuring the
% document.
%
%%%%%%%%%%%%%%%%%%%%%%%%%%%%%%%%%%%%%%%%%

%----------------------------------------------------------------------------------------
%	PACKAGES AND OTHER DOCUMENT CONFIGURATIONS
%----------------------------------------------------------------------------------------

\documentclass{resume} % Use the custom resume.cls style

\usepackage[left=0.75in,top=0.6in,right=0.75in,bottom=0.6in]{geometry} % Document margins
\usepackage[UTF8]{ctex}
\usepackage{graphicx}
\usepackage{wrapfig}
\newcommand{\tab}[1]{\hspace{.2667\textwidth}\rlap{#1}}
\newcommand{\itab}[1]{\hspace{0em}\rlap{#1}}
\begin{wrapfigure}{r}{70pt}
\vspace{-20pt}
\includegraphics[height=32.74mm,width=24.65mm]{photo.jpeg}
\end{wrapfigure}
\name{钱\quad{洋}} % Your name
\address{Blog | https://foxliveray.github.io/} % Your address
%\address{123 Pleasant Lane \\ City, State 12345} % Your secondary addess (optional)
\address{(+86)15088721518 \\ 275239328@qq.com} % Your phone number and email
\begin{document}

%----------------------------------------------------------------------------------------
%	EDUCATION SECTION
%----------------------------------------------------------------------------------------

\begin{rSection}{教育经历}

{\bf 西蒙弗雷泽大学, 加拿大温哥华} \hfill {\em 2020.09 - 2022.08} 
\\ 计算机科学与技术,硕士双学位\\
\\{\bf 浙江大学,浙江杭州} \hfill {\em 2019.09 - 2021.06} 
\\ 软件工程硕士,软件学院 \hfill {GPA:87.1/100 }
\\荣誉:2019-2020学年获优秀研究生\\
\\{\bf 北京交通大学,北京} \hfill {\em 2015.09 - 2019.06} 
\\ 软件工程学士,软件学院 \hfill {GPA:90.7/100|Rank:18/170 }
\\荣誉:2016-2017学年获校级优秀学生干部;3年均获得学习奖学金、社会工作奖学金;获校级大学生创新创业项目奖项
\end{rSection}

%----------------------------------------------------------------------------------------
%	WORK EXPERIENCE SECTION
%----------------------------------------------------------------------------------------

\begin{rSection}{工作经历}

\\{\bf 北京字节跳动网络有限公司}\hfill {\em 2020.05 - 2021.02}
\\EA BPM后端研发实习\\
项目描述:负责敏捷流程和事件平台的功能迭代,敏捷流程用于业务审批流程的自定义搭建和流转,事件平台用于解耦流程和节点的事件调用\\
\textbf{·} 负责敏捷版本迭代,包括表单发起条件和数据校验、扩展审批拒绝回退节点、优化表单数据Excel导出与表单版本更新逻辑、表单搜索控件外部数据源的集成,多入参数据源联动关系改造等,输出接入文档\\
\textbf{·} 负责扩展Open API接口功能,包括附件上传、流程数据和审批任务查询、支持发起接口审批人自选等\\
\textbf{·} 实现Open API基于接入应用AppId维度的鉴权,完善接口校验逻辑,文件处理类接口做跨域处理\\
\textbf{·} 基于业务方流程定制化事件平台开发,通过定时任务实现自动化;接入飞书开放平台定制bot消息和邮件发送;基于Redis实现了延迟任务队列,用于处理一些延迟事件调用的需求

\\{\bf 浙大区块链研究中心}\hfill {\em 2018.10 - 2019.03}
\\项目描述:与华为可靠性部门合作研究区块链在车联网系统中的可落地场景\\
\textbf{·} 阅读车联网相关论文,发现解决女巫攻击的业内方案存在CA服务器负载过大的问题\\
\textbf{·} 根据业内SCMS和C-ITS标准,基于PKI体系,设计区块链存储分发证书架构方案\\
\textbf{·} 利用Docker搭建Hyperleger Fabric区块链集群\\
\textbf{·} 调用Fabric Java和Go SDK实现车辆证书包的上链存储与下发\\
\textbf{·} 完成性能测试,FMEA失效模式分析和验证,公式算法替换测试

\\{\bf 君兰教育有限公司}\hfill {\em 2018.09 - 2018.12}
\\项目描述:辅助线下语言表演类课程,提供在线配套音视频材料学习、练习和录制功能的Android应用\\
\textbf{·} 负责基于MVP架构搭建Android项目开发框架\\
\textbf{·} 基于观察者模式思想,通过RxJava和Retrofit实现异步网络通信,完成与后端接口的交互\\
\textbf{·} 采用Google标准的Material Design实现统一风格的交互界面\\
\textbf{·} 完成1.0版本研发,包括用户基本信息模块、教材翻页浏览、视频教程播放、音视频录制上传等业务功能

\end{rSection}

%--------------------------------------------------------------------------------
%    Projects And Seminars
%-----------------------------------------------------------------------------------------------
\begin{rSection}{项目经历}
{\bf 基于区块链的文物管理系统}\\
\textbf{·} 基于SpringBoot与MyBatisPlus搭建项目开发框架,利用Docker做容器化部署,Nginx做负载均衡\\
\textbf{·} 基于Shiro安全框架实现用户RBAC权限控制,完成角色、菜单、权限管理模块的功能\\
\textbf{·} 基于趣链HyperChain区块链平台SDK对文物基础信息、交易信息、流转信息、出入境信息等上链存储,实现文物交易溯源业务功能

\end{rSection}
%----------------------------------------------------------------------------------------
%	TECHNICAL STRENGTHS SECTION
%----------------------------------------------------------------------------------------

\begin{rSection}{专业技能}
{\bf 后端研发为主,具备客户端开发能力}\\
\textbf{·} 数据结构、算法、计算机网络、操作系统基础较扎实\\
\textbf{·} 熟悉Java语法,IO,集合框架,多线程和并发编程知识\\
\textbf{·} 了解JVM内存模型、垃圾回收机制和类加载机制,有过一些GC异常排查和优化的实践\\
\textbf{·} 有一定的Python,PHP开发经验,Golang,C,C++编程经验\\
\textbf{·} 熟练使⽤Git⼯具进⾏协同开发,熟练使⽤Maven,Gradle等依赖构建⼯具\\ \textbf{·} 熟悉MySQL数据库和Redis⾮关系型数据库,了解HBase,Hadoop,Spark\\
\textbf{·} 熟练使⽤SpringBoot、MyBatis等开源框架,了解Docker容器化部署和K8S集群管理\\ 
\textbf{·} 了解Kafka,RocketMQ等消息中间件,使用RocketMQ完成业务功能解耦和异步\\
\textbf{·} 了解分布式理论、微服务架构和服务网格概念,了解Zookeeper,Thrift,Dubbo\\ 
\textbf{·} 熟悉常⽤的Linux命令,能够编写Shell脚本\\ 
\textbf{·} 熟悉CICD持续集成和持续部署的过程,了解DDD领域驱动设计思想和常用设计模式\\ 
\textbf{·} 具备良好的英语⽔平,能够阅读英文文献和文档资料,CET-6 : 546,托福 :92


\end{rSection}

\end{document}
